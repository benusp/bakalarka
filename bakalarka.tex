\documentclass[11pt,twoside,slovak,a4paper]{article}

\usepackage[slovak]{babel}
\usepackage[T1]{fontenc}
%\usepackage[IL2]{fontenc}
\usepackage{babel}
\usepackage[utf8]{inputenc}
\usepackage[nottoc]{tocbibind}
\usepackage{fancyhdr}
\usepackage{ifthen}
\usepackage{listings}
\usepackage{graphicx}
\usepackage{url} % príkaz \url na formátovanie URL
\usepackage{hyperref} % odkazy v texte budú aktívne (pri niektorých triedach dokumentov spôsobuje posun textu)

\usepackage{cite}
\usepackage{times}
\usepackage{comment}

\usepackage[dvips,dvipdfm,a4paper,centering,textwidth=14cm,top=4.6cm,headsep=.6cm,footnotesep=1cm,footskip=0.6cm,bottom=3.8cm]{geometry}
\usepackage{amsmath}
\usepackage{graphicx}

\usepackage{etoolbox}

\pagestyle{myheadings}

\begin{document}
	
	\begin{titlepage}
		\newlength{\myind}
		\addtolength{\myind}{57mm}
		
		\newlength{\myrulelength}
		\addtolength{\myrulelength}{.6\marginparwidth}
		\addtolength{\myrulelength}{\textwidth}
		\newcommand{\myrule}{\parbox{\myrulelength}{\hrulefill}}
		%\newcommand{\mybox}[1]{\hspace{.5\marginparwidth}\parbox{\myrulelength}{#1}}
		%\newcommand{\mycbox}[1]{\hspace{.5\marginparwidth}\parbox{\myrulelength}{\centering #1}}
		\newcommand{\mybox}[1]{\parbox{\myrulelength}{#1}}
		\newcommand{\mycbox}[1]{\parbox{\myrulelength}{\centering #1}}
		
		\centering
		\large 		Slovenská technická univerzita v Bratislave\\
		Fakulta informatiky a informačných technológií\\
		Študijný program: Informatika
		
		\vspace{4mm}
		\myrule
		
		\vspace{55mm}
		
		\mycbox{Peter Beňuš}
		
		\vspace{10mm}
		
		\mycbox{\LARGE \textbf{Automatické testovanie softvéru}}
		
		\vspace{4mm}
		\mycbox{\large Bakalársky projekt 1}
		
		
		\vfill
		\begin{flushleft}
			Vedúci bakalárskeho projektu: Ing. Karol Rástočný\\
			December 2015
		\end{flushleft}
	\end{titlepage}
	
	
%	\begin{titlepage}
%		\section*{Anotácia}
%		\ldots
%	\end{titlepage}
	
	
	\tableofcontents
	
	
	\pagestyle{fancy}
	\headheight 14pt
	
	\section{Testovanie softvéru}
	Testovanie softvéru je empirická činnosť, ktorá skúma kvalitu testovaného produktu alebo služby vykonávaná na podanie informácií o kvalite všetkým zainteresovaným osobám\cite{Kaner2006}. V súčastnosti existuje veľa spôsobov testovania a veľa častí životného cyklu softvéru v ktorých sa aplikujú iné typy testov. Testovanie softvéru môžem rozdeliť na kategórie podľa postupu, ktorý sa používa pri testovaní, podľa spôsobu testovanie a podľa úrovne testu.
	
	\subsection{Podľa postupu testovania}
	\begin{itemize}
		\item \textbf{Testovanie formou čiernej skrinky} \newline
			Testuje funkcionalitu bez informácií o tom ako je softvér implementovaný. Tester dostane iba informácie o tom, aký by mal byť výsledok testu po zadaní vstupných dát a kontroluje výstup softvéru či sú výstupné dáta totožné s očakávanými\cite{EST2002}. Test je konštruovaný z funkcionálnych vlastností, ktoré sú špecifikované požiadavkách na program\cite{Moha1991}.  Výhodou tohto typu testovania je, že tester nie je ovplyvnený štruktúrou zdrojového kódu a tým môže odhaliť chyby aj tam, kde to programátor nehľadal lebo to považoval za správne pri pohľade na zdrojový kód, ale bola tam chyba, ktorá zo zdrojového kódu nemusí byť viditeľná (napríklad nesprávne, resp. nedostatočné ošetrenie nekorektných vstupov). Hlavnou výhodou je, že tester nemusí poznať zdrojový kód a preto môže oveľa rýchlejšie vytvoriť testy. Nevýhodou je úroveň otestovania systému, pretože tvorca testov nevie ako program funguje, a preto veľmi pravdepodobne nebude schopný vytvoriť test, ktorý by testoval všetky vetvy programu. Používa sa pri jednotkovom, integračnom, systémovom a akceptačnom testovaní. Okrem použitia pri rôznych úrovní testov sa využíva aj na validáciu softvéru\cite{EST2002}.
		\item \textbf{Testovanie formou bielej skrinky} \newline
			Pri tomto spôsobe testovania je testerovi známa vnútorná štruktúra softvéru a aj konkrétna implementácia. Test sa tvorí tak, aby bola otestovaná každá vetva zdrojového kódu\cite{EST2002}. Používa sa pri jednotkovom testovaní na skoré odhalenie všetkých chýb, pri integračnom testovaní na testovanie správnej spolupráce rôznych jednotiek programu a aj pri regresnom testovaní, kde sa používajú recyklované testovacie prípady z integračného a jednotkového testovania a taktiež slúži na verifikáciu. Výhodou je schopnosť otestovať komplexne všetky vetvy, ktoré program na danej úrovni testu vykonáva, ale nevýhodou je, že tester musí mať dobré vedomosti o zdrojovom kóde a v niektorých prípadoch tvorenia testov môže byť znalosť zdrojového kódu nevýhoda.
		\item \textbf{Testovanie formou sivej skrinky} \newline
			Spôsob testovania, pri ktorom je známy zdrojový kód (nemusí byť sprístupnený úplne celý), ale testy sa vykonávajú rovnako ako pri testovaní formou čiernej skrinky. Používa sa napríklad pri integračnom testovaní ak máme dva moduly od rôznych vývojárov a odkryté sú len rozhrania \cite{EST2002}. Poskytuje výhody obidvoch predchádzajúcich prístupov, ale má oproti nim aj nejaké nevýhody. Oproti testovaniu čiernou skrinkou má výhodu v lepšom pokrytí rôznych vetiev zdrojového kódu, ale je časovo náročnejšie na tvorbu testov. Oproti testovaniu bielou skrinkou je menej náročné na znalosť zdrojového kódu, pretože ho nemá prístupný celý, ale nepokrýva všetky vetvy programu a preto je menej komplexné.
	\end{itemize}
	
	\subsection{Podľa spôsobu testovania}
	\begin{itemize}
		\item \textbf{Statické testovanie} \newline
			Statické testovanie je často implicitné. Zahŕňa napríklad kontrolu zdrojového kódu programátorom jeho čítaním hneď po napísaní, kontrolu štruktúry a syntaxe kódu nástrojom alebo editorom, v ktorom sa zdrojový kód píše. Program nie je potrebné spúšťať, ale analýza zdrojového kódu založená na upravovacích pravidlách zistí v zdrojovom kóde rôzne možné chyby, ktoré sa zvyčajne objavujú v spravovaní pamäte, neinicializovaných premenných, výnimke nulového smerníku, porušení prístupu k poľu a taktiež pretečení vyrovnávacej pamäti\cite{Wei2014}.
		\item \textbf{Dynamické testovanie} \newline
			Dynamické testovanie prebieha už na spustenom softvéri. Začína skôr ako je úplne dokončený softvér, pretože sa počas vývoja testujú aj menšie spustiteľné časti. K dynamickému testovaniu sa viaže validácia.
	\end{itemize}
	
	\subsection{Podľa úrovne testu}
	\begin{itemize}
		\item \textbf{Jednotkové testovanie} \newline
			Jednotkové testovanie je metóda testovania softvéru, pri ktorej sa testujú individuálne komponenty (jednotky) zdrojového kódu. Zvyčajne nie je testovacou fázou v zmysle nejakého obdobia na tvorbe projektu, ale skôr je to posledný krok písania časti zdrojového kódu\cite{Alba2008}. Programátori takmer vykonávajú jednotkové testovanie takmer stále, či už pri testovaní vlastného zdrojového kódu alebo kódu iného programátora\cite{Alba2008}. Kvalitné testovanie na tejto úrovni môže výrazne znížiť cenu a čas potrebný na vývoj celého softvéru\cite{EST2002}.
		\item \textbf{Testovanie komponentov} \newline
			Počas testovania komponentov sa testeri zameriavajú na chyby v ucelených častiach systému. Vykonávanie testu zvyčajne začína, keď je už prvý komponent funkčný spolu so všetkým potrebným (napr. ovládače) na fungovanie tohto komponentu bez zbytku systému\cite{Alba2008}.\newline
			Testovanie komponentov má sklon viezť k štrukturálnemu testovaniu alebo testovaniu formou bielej skrinky. Ak je komponent nezávislý môže sa použiť aj testovanie formou čiernej skrinky\cite{Alba2008}.
		\item \textbf{Integračné testovanie} \newline
			V integračnom testovaní sa testeri zameriavajú na hľadanie chýb vo vzťahoch a rozhraniach medzi pármi a skupinami komponentov. Integračné testovanie musí byť koordinované, aby sa správna množina komponentov spojila správnym spôsobom a v správnom čase	pre najskoršie možné odhalenie integračných chýb\cite{Alba2008}.
			Niektoré projekty nepotrebujú formálnu fázu integračného testovania. Ak je projekt množinou nezávislých aplikácií, ktoré nezdieľajú dáta alebo sa nespúšťajú navzájom, môže byť táto fáza preskočená\cite{Alba2008}.
		\item \textbf{Systémové testovanie} \newline		
			Systémove testovanie je vykonávané na úplnom a integrovanom systéme za účelom vyhodnotenia súladu systému z jeho špecifikovanými požiadavkami\cite{Dictionary}. 
			Niekedy, napríklad pri testovaní inštalácie a použiteľnosti, sa tieto testy pozerajú na systém z pohľadu zákazníka alebo koncového používateľa. Inokedy sú testy zdôrazňujú konkrétne aspekty, ktoré môžu byť nepovšimnuté používateľom, ale kritické pre správne fungovanie systému \cite{Alba2008}.
		\item \textbf{Akceptačné testovanie} \newline		
			Akceptačné testovanie je formálne testovanie zamerané na potreby používateľa, požiadavky a  biznis procesy vedúce k rozhodnutiu či systému vyhovuje alebo nevyhovuje akceptačným kritériám a umožniť používateľovi, zákazníkovi alebo inému splnomocnenému subjektu či má alebo nemá byť systém akceptovaný\cite{Veenendaal2010}. \newline
			Narozdiel od predchádzajúcich foriem testovania, akceptačné testovanie demonštruje, že systém spĺňa požiadavky \cite{Alba2008}. \newline
			V komerčnej sfére sú niekedy tieto testy nazývané aj podľa toho kým sú vykonávané "alfa testy" (používateľmi vo firme) alebo "beta testy" (súčasnými alebo potenciálnymi zákazníkmi) \cite{Alba2008}.
	\end{itemize}
	
	\section{xUnit}
		xUnit je označenie pre skupinu frameworkov, ktoré slúžia na jednotkové testovanie. Vznikol pôvodne pre programovací jazyk Smalltalk a veľmi rýchlo sa stal známym a úspešným. Dnes už majú všetky bežne používané programovacie jazyky minimálne jeden vlastný framework na jednotkové testovanie a mnoho z nich je odvodených práve od xUnit. 		http://www.martinfowler.com/bliki/Xunit.html \newline

		Spoločné znaky frameworkov patriacich do skupinu xUnit:
		\begin{itemize}
			\item \textbf{Test runner} Je to spustitešný program, ktorý vykoná test a zároveň vytvorí správu o výsledku testu.
			\item \textbf{Test case} Je to základná trieda, od ktorej sú odvodené všetky testy.
			\item \textbf{Test fixtures} Množina podmienok definovaných programátorom, ktoré musia byť splnené pred vykonaním testu. Po teste by mali byť vrátené do pôvodného stavu.
			\item \textbf{Test suites} Množina testov, ktoré zdieľajú podmienky potrebné pre spustenie testu.
			\item \textbf{Vykonanie testu} Vykonanie individuálneho jednotkového testu.
			\item \textbf{Test result formatter} Výsledok testu môže byť v jednom alebo viacerých formátoch. Okrem textu čitateľného pre človeka sa často používa aj výstup vo formáte XML.
			\item \textbf{Assertion} Je to funkcia alebo makro, ktorá definuje stav testovanej jednotky. Zvyčajne je to logická podmienka, ktorá pravdivá ak je výsledok testu správny. Zlyhanie väčšinou končí volaním výnimky, ktorá ukončí vykonávanie testu.
		\end{itemize}
	
	
	\bibliography{literatura}
	\bibliographystyle{plain}
\end{document}
